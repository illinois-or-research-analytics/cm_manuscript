\documentclass[12pt, oneside]{article}   	% use "amsart" instead of "article" for AMSLaTeX format
\usepackage{textcomp}
\usepackage{geometry}                		% See geometry.pdf to learn the layout options. There are lots.
\geometry{letterpaper}                   		% ... or a4paper or a5paper or ... 
%\geometry{landscape}                		% Activate for rotated page geometry
%\usepackage[parfill]{parskip}    		% Activate to begin paragraphs with an empty line rather than an indent
\usepackage{graphicx}				% Use pdf, png, jpg, or eps§ with pdflatex; use eps in DVI mode
\usepackage{caption}
\usepackage{subcaption}								% TeX will automatically convert eps --> pdf in pdflatex		
\usepackage{color}
\usepackage{amssymb}
\usepackage{amsthm}
\usepackage{url}
\newtheorem{theorem}{Theorem}
\newtheorem{definition}{Definition}
\usepackage{natbib}
\usepackage{xcolor}
% removed hyperref because of arXiv complaining
\usepackage{authblk}
\usepackage{float}
\usepackage{rotating}
\usepackage{adjustbox}
\usepackage[font=small,labelfont=bf]{caption}
%\usepackage{changes}
\usepackage{changes}
\definechangesauthor[name={George Chacko}, color=blue]{gc}

\usepackage{authblk}
\title{Supplementary Material For ``Finding Well-Connected Communities in Real-World Networks''}
\author[1]{Minhyuk Park\thanks{author order to be determined later}}
\author[1]{Yasamin Tabatabaee\thanks{author order to be determined later}}
\author[1]{Baqiao Liu\thanks{author order to be determined later}}
\author[1]{Placeholder1\thanks{author order to be determined later}}
\author[1]{Placeholder2\thanks{author order to be determined later}}
\author[1]{Placeholder3\thanks{author order to be determined later}}
\author[2]{Dmitriy Korobskiy}
\author[1,3]{George Chacko\thanks{chackoge@illinois.edu}}
\author[1]{Tandy Warnow\thanks{warnow@illinois.edu}}
\affil[1]{Department of Computer Science, University of Illinois Urbana-Champaign, Urbana, IL 61801}
\affil[2]{NTT DATA, McLean, VA 22102}
\affil[3]{Office of Research, Grainger College of Engineering, University of Illinois Urbana-Champaign, Urbana, IL 61801}


\begin{document}
\maketitle

\section{}

\section{CM Pipeline} 

The pipeline consists of stages
\begin{itemize}
\item Clustering an input network using either the Leiden algorithm or Iterative K-core Clustering (IKC)
\item Removing clusters of size $<$ 10a s well as trees ($n=m+1$). Depending on the user and the dataset being analyzed one of the following
options was used. 
\begin{enumerate}
\item belinda \citep{belinda2022} expressions, e.g., $g1.filter(pl.col('n') > 10)$ and \\ $g1.filter(pl.col('n') != pl.col('m') + 1)$
\item find\_trees.py- a slow script that relies on Networkit \citep{Staudt2016}
\item cluster\_analyzer.py- a parallelized version of find\_trees.py
\item subset\_graph\_nonnetworkit.R- a fast script without dependence on Networkit that also returns maxDegree for a cluster. 
\end{enumerate}

\item Applying connectivity modifier \citep{cm2022}
\item Removing clusters of size 10 and below as well as trees ($n=m+1$)
\end{itemize}
\section{}

\bibliographystyle{apalike}
\bibliography{cmv1}
\end{document}  